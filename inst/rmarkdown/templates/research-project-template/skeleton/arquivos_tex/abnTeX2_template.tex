% ---------------------------------------------------------------------------------------
% Definição do estilo de documento
% ---------------------------------------------------------------------------------------
\documentclass[
	12pt,       % tamanho da fonte
  %======================================================================================
  $if(book_format)$
  	% Para encadernação %
    openright, % Capítulos em página ímpar(insere as páginas em branco)
  	twoside,   % Páginas pares tem a margem direita de 3cm e esquerda de 2cm
  	%------------------------------------------------------------------------------------
  $else$
  	% Para pdf online
  	oneside,    % Todas páginas tem a margem esquerda de 3cm e direita de 2cm
  	openany,    % Capítulos independente da página ser par ou ímpar
  $endif$
  %======================================================================================
  % a4paper,	% tamanho do papel.
	english,	% idioma adicional para hifenização
	french,		% idioma adicional para hifenização
	spanish,	% idioma adicional para hifenização
	brazil,		% o último idioma é o principal do documento
	]{abntex2}
% ---------------------------------------------------------------------------------------

% ---------------------------------------------------------------------------------------
% Pacotes utilizados
% ---------------------------------------------------------------------------------------
\usepackage{longtable,booktabs}
\usepackage{lmodern}			% Usa a fonte Latin Modern
\usepackage[T1]{fontenc}		% Selecao de codigos de fonte.
\usepackage[utf8]{inputenc}		% Codificacao do documento (conversão automática dos acentos)
\usepackage{indentfirst}		% Indenta o primeiro parágrafo de cada seção.
\usepackage{xcolor}				% Controle das cores
\usepackage{graphicx}			% Inclusão de gráficos
\usepackage{microtype} 			% Para melhorias de justificação
\usepackage{color}              % Cores
\usepackage{fancyvrb}           % Ambientes de código
\usepackage{mdframed} %nice frames
\usepackage{float}
\usepackage{pgffor}
% ---------------------------------------------------------------------------------------

% ---------------------------------------------------------------------------------------
% Pacotes de citações
% ---------------------------------------------------------------------------------------
\usepackage[brazilian,hyperpageref]{backref}	 % Paginas com as citações na bibl
\usepackage[alf]{abntex2cite}	% Citações padrão ABNT
% ---------------------------------------------------------------------------------------

% ---------------------------------------------------------------------------------------
% Comandos e parâmetros renomeados na configuração de pacotes
% ---------------------------------------------------------------------------------------
\newcommand{\VerbBar}{|}
\newcommand{\VERB}{\Verb[commandchars=\\\{\}]}
\DefineVerbatimEnvironment{Highlighting}{Verbatim}{commandchars=\\\{\}}

\definecolor{Function}{HTML}{043C5E}
\definecolor{Param}{HTML}{10AFF2}
\definecolor{Comment}{HTML}{6E6E6E}
\definecolor{Operator}{HTML}{00838f}
\definecolor{String}{HTML}{009658}
\definecolor{Value}{HTML}{00ABA2}
\definecolor{Par}{HTML}{9897C7}
\definecolor{bg}{HTML}{E8E8E8}

\newenvironment{Shaded}{\begin{snugshade}}{\end{snugshade}}

\newcommand{\FunctionTok}[1]{\textcolor{Function}{\textbf{#1}}}
\newcommand{\ParamTok}[1]{\textcolor{Param}{\textbf{#1}}}
\newcommand{\CommentTok}[1]{\textcolor{Comment}{\textit{#1}}}
\newcommand{\OperatorTok}[1]{\textcolor{Operator}{#1}}
\newcommand{\StringTok}[1]{\textcolor{String}{#1}}
\newcommand{\ValueTok}[1]{\textcolor{Value}{#1}}
\newcommand{\ParTok}[1]{\textcolor{Par}{#1}}
\newcommand{\NormalTok}[1]{#1}

\mdfdefinestyle{rcode}
{
backgroundcolor=bg,
roundcorner=10pt,
innertopmargin=10pt,
innerbottommargin=10pt,
innerrightmargin=10pt,
innerleftmargin=10pt,
leftmargin = 0.5pt,
rightmargin = 1pt,
linewidth=0.5pt
}

\renewcommand{\arraystretch}{1.5}
% Configurações do pacote backref
% Usado sem a opção hyperpageref de backref
\renewcommand{\backrefpagesname}{Citado na(s) página(s):~}
% Texto padrão antes do número das páginas
\renewcommand{\backref}{}
% Define os textos da citação
\renewcommand*{\backrefalt}[4]{
	\ifcase #1 %
		Nenhuma citação no texto.%
	\or
		Citado na página #2.%
	\else
		Citado #1 vezes nas páginas #2.%
	\fi}%

\renewcommand{\imprimircapa}{%
  \begin{capa}%
    \center
    \ABNTEXchapterfont\large\imprimirautor

    \vfill
    \begin{center}
    \ABNTEXchapterfont\bfseries\LARGE\imprimirtitulo
    \end{center}
    \begin{center}
    \ABNTEXchapterfont\large{$subtitulo$}
    \end{center}
    \vfill

    \large\imprimirlocal

    \large\imprimirdata

    \vspace*{1cm}
  \end{capa}
}
% ---------------------------------------------------------------------------------------


% ---------------------------------------------------------------------------------------
% Informações de dados para CAPA e FOLHA DE ROSTO
% ---------------------------------------------------------------------------------------
\titulo{$titulo$}
\autor{$autor$}
\local{$local$}
\data{$data$}
\preambulo{$preambulo$}
% ---------------------------------------------------------------------------------------


% ---------------------------------------------------------------------------------------
% Configurações de aparência do PDF final
% ---------------------------------------------------------------------------------------
% Alterando o aspecto da cor azul
\definecolor{blue}{RGB}{41,5,195}

% Informações do PDF
\makeatletter
\hypersetup{
     	%pagebackref=true,
		pdftitle={\@title},
		pdfauthor={\@author},
    	pdfsubject={\imprimirpreambulo},
	    pdfcreator={LaTeX with abnTeX2},
		pdfkeywords={abnt}{latex}{abntex}{abntex2}{projeto de pesquisa},
		colorlinks=true,       		% false: boxed links; true: colored links
    	linkcolor=blue,          	% color of internal links
    	citecolor=blue,        		% color of links to bibliography
    	filecolor=magenta,      		% color of file links
		urlcolor=blue,
		bookmarksdepth=4
}
\makeatother
% ---------------------------------------------------------------------------------------

% ---------------------------------------------------------------------------------------
% Espaçamentos entre linhas e parágrafos
% ---------------------------------------------------------------------------------------
% O tamanho do parágrafo é dado por:
\setlength{\parindent}{1.3cm}

% Controle do espaçamento entre um parágrafo e outro:
\setlength{\parskip}{0.2cm}  % tente também \onelineskip
% ---------------------------------------------------------------------------------------

% ---------------------------------------------------------------------------------------
% Compila o indice
% ---------------------------------------------------------------------------------------
\makeindex
% ---------------------------------------------------------------------------------------

% ---------------------------------------------------------------------------------------
% Início do documento
% ---------------------------------------------------------------------------------------
\begin{document}

% Seleciona o idioma do documento (conforme pacotes do babel)
%\selectlanguage{english}
\selectlanguage{brazil}

% Retira espaço extra obsoleto entre as frases.
\frenchspacing

% ---------------------------------------------------------------------------------------
% Capa
% ---------------------------------------------------------------------------------------
\imprimircapa
% ---------------------------------------------------------------------------------------

% ---------------------------------------------------------------------------------------
% Folha de rosto
% ---------------------------------------------------------------------------------------
\imprimirfolhaderosto
% ---------------------------------------------------------------------------------------

% ---------------------------------------------------------------------------------------
% Inserir lista de ilustrações
% ---------------------------------------------------------------------------------------
\pdfbookmark[0]{\listfigurename}{lof}
\listoffigures*
\cleardoublepage
% ---

% ---------------------------------------------------------------------------------------
% Inserir lista de tabelas
% ---------------------------------------------------------------------------------------
\pdfbookmark[0]{\listtablename}{lot}
\listoftables*
\cleardoublepage
% ---------------------------------------------------------------------------------------

% ---------------------------------------------------------------------------------------
% Inserir lista de abreviaturas e siglas
% ---------------------------------------------------------------------------------------
 \begin{siglas}
  $for(lista-abreviaturas-e-siglas/pairs)$
    \item[$lista-abreviaturas-e-siglas.key$] $lista-abreviaturas-e-siglas.value$
  $endfor$
 \end{siglas}
% ---------------------------------------------------------------------------------------

% ---------------------------------------------------------------------------------------
% inserir o sumario
% ---------------------------------------------------------------------------------------
\begin{sffamily}
\noindent\normalsize{\textbf{Resumo}}

\noindent\small{$resumo$}

\noindent\normalsize{\textbf{Palavras-chave: }}\small{$palavras-chave$}
\end{sffamily}
\pdfbookmark[0]{\contentsname}{toc}
\tableofcontents*
\cleardoublepage
% ---------------------------------------------------------------------------------------
\textual
% ---------------------------------------------------------------------------------------

$body$

% ---------------------------------------------------------------------------------------
% Referências bibliográficas
% ---------------------------------------------------------------------------------------
\bibliography{arquivos_tex/references}
% ---------------------------------------------------------------------------------------

% ---------------------------------------------------------------------------------------
% INDICE REMISSIVO
% ---------------------------------------------------------------------------------------
\phantompart
\printindex
% ---------------------------------------------------------------------------------------

\end{document}
